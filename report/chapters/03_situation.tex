%!TEX root = ../main.tex

\chapter{Technology}
\label{chp:Technology}

The existing technological infrastructure provides a robust starting point for our GIS application development. The municipality currently utilizes PostgreSQL with the PostGIS extension for managing a comprehensive topographic database at a 1:2000 scale. This database, designed in alignment with national standards and maintaining an NC1 level of detail, is a fundamental repository of spatial information. Additionally, Geoserver serves as the platform for sharing and distributing geographical data, ensuring the availability of this critical data to relevant stakeholders. Building upon this established foundation, our application will leverage the PostgreSQL-PostGIS combination for efficient data storage, querying, and manipulation. The familiarity with these technologies will expedite the integration of our GIS application while enhancing data accuracy and accessibility.

\section{Existing Data}

Central to our project is the utilization of existing geographical data. The topographic database, meticulously constructed in the ETRF2000 reference system, presents a comprehensive spatial representation of the municipality. This dataset, offering a 1:2000 scale, encompasses intricate details and geometries crucial for accurate plan representation. Additionally, the availability of the current Master Plan and zone geometries in shapefile format, also aligned with the ETRF2000 reference system, furnishes us with foundational spatial information. This existing data will serve as a cornerstone for our application's visualization and manipulation features. Leveraging shapefiles ensures compatibility and consistency, allowing us to seamlessly integrate these resources into our system's functionalities.

\section{Market Analysis}

A thorough market analysis reveals a distinct gap in available solutions that cater to the specific requirements of managing Master Plans and their variants. Packaged software solutions fail to align with the intricacies of Fano's needs, prompting the development of a custom GIS application. 

In summary, the starting situation is characterized by a well-established technological foundation, accessible data resources, and an undeniable market need for a comprehensive GIS solution. Building upon these strengths, our GIS application will harness existing technologies and data to deliver a tailored solution that meets the intricate requirements of managing the Master Plan and its variants while filling a void in the available software landscape.