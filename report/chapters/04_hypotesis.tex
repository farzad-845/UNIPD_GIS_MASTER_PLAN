%!TEX root = ../main.tex

\chapter{Working Hypotheses}
\label{ch:working_hypotheses}

The working hypothesis is that a web-based GIS application can be developed to meet the requirements for master plan viewing and querying. This application will leverage the municipality's existing GeoServer, PostgreSQL, and PostGIS components to serve the master plan data through a web mapping interface. Open-source JavaScript libraries like OpenLayers or Leaflet can be used for the interactive map client. The application server can connect to the PostGIS database to retrieve master plan data filtered by parcel ID for zoning lookups.

Additionally, an editing plugin can be developed for the open source OpenJUMP GIS desktop application. This would allow municipal technicians to create and edit new master plan variant geometries and attributes. The plugin can connect to the PostGIS database to load existing plan data, and save any edits made back to the database. Integrating these two components - the web portal and OpenJUMP plugin - will provide both public viewing and internal editing capabilities for managing master plan data.