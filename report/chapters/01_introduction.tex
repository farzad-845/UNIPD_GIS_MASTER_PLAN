%!TEX root = ../main.tex

\chapter{Introduction}
\label{chp:Introduction}

The primary objective of this project is to develop an advanced Geographic Information System (GIS) application tailored for the management and utilization of the Master Plan of the Italian province of Fano. With a population of approximately 250,000 inhabitants, Fano's Master Plan serves as a crucial blueprint for urban and territorial development. This project seeks to address the specific needs of planners and the general public, offering them a platform that facilitates the creation, maintenance, and accessibility of plan variants, thus fostering efficient urban planning and informed decision-making.

In essence, this GIS application aims to streamline the process of managing the Master Plan and its subsequent variants. By providing planning technicians with an intuitive and feature-rich environment, the application will enable them to create and manipulate plan variants seamlessly. This includes the ability to define new zones, allocate attributes indicating intended land use (such as residential, agricultural, public green, commercial, and industrial), and transition between various plan variant statuses (in progress, planned, in force, not approved) while recording relevant dates for each change. Simultaneously, the application will empower citizens to effortlessly access the Master Plan and its associated variants. Citizens will be able to visualize zoning information, identify land parcels using cadastral identification codes, and obtain printable documents certifying the permissible land uses based on the plan's regulations.
